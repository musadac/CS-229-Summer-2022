\item \points{15} {\bf KL divergence and Maximum Likelihood}

The Kullback-Leibler (KL) divergence is a measure of how much
one probability distribution is different from a second one. It is a concept that originated in Information Theory, but has made its
way into several other fields, including Statistics, Machine Learning,
Information Geometry, and many more. In Machine Learning, the KL
divergence plays a crucial role, connecting various concepts
that might otherwise seem unrelated.

In this problem, we will introduce KL divergence over discrete
distributions, practice some simple manipulations, and see its
connection to Maximum Likelihood Estimation.

The \emph{KL divergence} between two discrete-valued
distributions $P(X), Q(X)$  over the outcome space $\mathcal{X}$ is defined as follows\footnote{If $P$ and $Q$ are densities for continuous-valued random variables, then the sum is replaced by an integral, and everything stated in this problem works fine as well.  But for the sake of simplicity, in this problem we'll just work with this form of KL divergence for probability mass functions/discrete-valued distributions.}:

$$\KL(P\|Q) = \sum_{x \in \mathcal{X}}P(x)\log\frac{P(x)}{Q(x)}$$

For notational convenience, we assume $P(x)>0, \forall x$.
(One other standard thing to do is to adopt the convention that
``$0 \log 0 = 0$.'')  Sometimes, we also write the KL divergence more explicitly as
$\KL(P||Q) = \KL(P(X)||Q(X))$.

\emph{Background on Information Theory}

Before we dive deeper, we give a brief (optional) Information Theoretic
background on KL divergence. While this introduction is not necessary
to answer the assignment question, it may help you better understand
and appreciate why we study KL divergence, and how Information Theory
can be relevant to Machine Learning.

We start with the \emph{entropy} $H(P)$ of a probability distribution $P(X)$, which is defined as

$$ H(P) = -\sum_{x \in \mathcal{X}} P(x) \log P(x). $$

Intuitively, entropy measures how dispersed a probability
distribution is. For example, a uniform distribution is considered to have
very high entropy (i.e. a lot of uncertainty), whereas a distribution that assigns
all its mass on a single point is considered to have zero entropy (i.e. no uncertainty). Notably, it can be shown that among continuous distributions over $\R$, the Gaussian
distribution $\mathcal{N}(\mu,\sigma^2)$ has the highest entropy (highest uncertainty) among all possible distributions that have the given mean $\mu$ and variance $\sigma^2$.

To further solidify our intuition, we present motivation from communication theory. Suppose we want to communicate from a source to a destination, and our messages are always (a sequence of) discrete symbols over space $\mathcal{X}$ (for example, $\mathcal{X}$ could be letters $\{ \text{a}, \text{b}, \dots, \text{z} \}$).
We want to construct an encoding scheme for our symbols in the form of sequences of binary bits
that are transmitted over the channel. Further, suppose that in the long run the
frequency of occurrence of symbols follow a probability distribution $P(X)$. This
means, in the long run, the fraction of times the symbol $x$ gets transmitted is $P(x)$.

A common desire is to construct an encoding scheme such that the average number of bits
per symbol transmitted remains as small as possible. Intuitively, this means we want very frequent
symbols to be assigned to a bit pattern having a small number of bits. Likewise, because we
are interested in reducing the average number of bits per symbol in the long term,
it is tolerable for infrequent words to be assigned to bit patterns having a
large number of bits, since their low frequency has little
effect on the long term average. The encoding scheme can be as complex as we desire, for example,
a single bit could possibly represent a long sequence of multiple symbols (if
that specific pattern of symbols is very common). The entropy of a probability distribution
$P(X)$ is its optimal bit rate, i.e., the lowest average bits per message that can possibly be achieved if
the symbols $x \in \mathcal{X}$ occur according to $P(X)$. It
does not specifically tell us \emph{how} to construct that optimal encoding scheme. It only
tells us that no encoding can possibly give us a lower long term bits per message
than $H(P)$.

To see a concrete example, suppose our messages have a vocabulary of $K=32$ symbols,
and each symbol has an equal probability of transmission in the long term (i.e, uniform
probability distribution). An encoding scheme that
would work well for this scenario would be to have $\log_2 K$ bits per symbol, and assign
each symbol some unique combination of the $\log_2 K$ bits. In fact, it turns out that
this is the most efficient encoding one can come up with for the uniform distribution
scenario.

It may have occurred to you by now that the long term average number of bits per message depends only
on the frequency of occurrence of symbols. The encoding scheme of scenario A can in theory be
reused in scenario B with a different set of symbols (assume equal vocabulary size for simplicity),
with the same long term efficiency, as long as the symbols of scenario B follow the same probability
distribution as the symbols of scenario A. It might also have occurred to you, that reusing the encoding scheme
designed to be optimal for scenario A, for messages in scenario B having a \emph{different probability} of symbols, will always
be suboptimal for scenario B. To be clear, we do not need know \emph{what} the specific optimal
schemes are in either scenarios. As long as we know the distributions of their symbols,
we can say that the optimal scheme designed for scenario A will be suboptimal for scenario B if the
distributions are different.

Concretely, if we reuse the optimal scheme designed for a scenario having symbol distribution $Q(X)$,
into a scenario that has symbol distribution $P(X)$, the long term average
number of bits per symbol achieved is called the \emph{cross entropy}, denoted by $H(P, Q)$:

$$H(P, Q) = -\sum_{x \in \mathcal{X}} P(x) \log Q(x). $$

To recap, the entropy $H(P)$ is the best possible long term average bits per message (optimal) that
can be achieved under a symbol distribution $P(X)$ by using an encoding scheme (possibly unknown)
specifically designed for $P(X)$. The cross entropy $H(P, Q)$ is the long term average bits per
message (suboptimal) that results under a symbol distribution $P(X)$, by reusing an encoding
scheme (possibly unknown) designed to be optimal for a scenario with symbol distribution $Q(X)$.

Now, KL divergence is the penalty we pay, as measured in average number of bits, for using the
optimal scheme for $Q(X)$, under the scenario where symbols are actually distributed as $P(X)$. It is
straightforward to see this

\begin{align*}
\KL(P \| Q) &= \sum_{x \in \mathcal{X}} P(x) \log\frac{P(x)}{Q(x)} \\
&= \sum_{x \in \mathcal{X}} P(x) \log P(x) - \sum_{x\in \mathcal{X}}P(x) \log Q(x) \\
&= H(P, Q) - H(P). \quad \text{(difference in average number of bits.)}
\end{align*}

If the cross entropy between $P$ and $Q$ is $H(P)$ (and hence $\KL(P||Q) = 0$) then it necessarily
means $P = Q$. In Machine Learning, it is a common task to find a distribution $Q$ that is ``close'' to
another distribution $P$. To achieve this, it is common to use $\KL(Q||P)$ as the loss function to be optimized.
As we will see in this question below, Maximum Likelihood Estimation, which is a commonly used
optimization objective, turns out to be equivalent to minimizing the KL divergence between the training data
(i.e. the empirical distribution over the data)
and the model.

Now, we get back to showing some simple properties of KL divergence.

\begin{enumerate}

  \item\subquestionpoints{5} \textbf{Nonnegativity.}

Prove the following:
$$\forall P, Q. \quad \KL(P\|Q) \geq 0$$

and

\[
\KL(P\|Q) = 0 \quad\quad\text{ if and only if }\quad\quad P=Q.
\]

[Hint: You may use the following result, called {\bf Jensen's inequality}.  If $f$ is a convex function, and $X$ is a random variable, then $E[f(X)] \geq f(E[X])$. Moreover, if $f$ is strictly convex ($f$ is convex if its Hessian satisfies $H \geq 0$; it is \emph{strictly} convex if $H > 0$; for instance $f(x) = -\log x$ is strictly convex), then $E[f(X)] = f(E[X])$ implies that $X=E[X]$ with probability 1; i.e., $X$ is actually a constant.]


\ifnum\solutions=1 {
  \newcommand\tab[1][1cm]{\hspace*{#1}}
\begin{answer}
\\ \\
\tab $D_{KL}(P||Q) = -\sum_x P(x)log\frac{P(x)}{Q(x)}$ \\ \\
\tab[3cm] $  = \sum_x P(x)log\frac{P(x)}{Q(x)}$ \\  \\
\tab[3cm] $  = log \sum_x P(x)\frac{Q(x)}{P(x)}$ \\ \\
\tab[3cm] $  = log \sum_x Q(x)$ \\ \\
\tab[3cm] $  = log \ 1$ \\ \\
\tab[3cm] $   = 0$ \\ \\
As we have equality if and only if $\frac{Q(x)}{P(x)} = E_p[\frac{Q(x)}{P(x)}]$  and  P(x) = Q(x) Using Jensen's inequality. \\ \\ 
So \\ \\
\tab $D_{KL}(P||Q) = 0$ \\ \\
\end{answer}

} \fi

  \item \subquestionpoints{5} \textbf{Chain rule for KL divergence.}

The KL divergence between 2
conditional distributions $P(X|Y),Q(X|Y)$ is defined as follows:
\[
\KL(P(X|Y)\|Q(X|Y))
= \sum_y P(y) \left(\sum_x P(x|y)\log\frac{P(x|y)}{Q(x|y)}\right)
\]
This can be thought of as the expected KL divergence between the
corresponding conditional distributions on $x$ (that is, between
$P(X|Y=y)$ and $Q(X|Y=y)$), where the expectation is taken over the
random $y$.

Prove the following chain rule for KL divergence:
$$ \KL(P(X,Y)\|Q(X,Y)) = \KL(P(X)\|Q(X)) + \KL(P(Y|X)\|Q(Y|X)). $$


\ifnum\solutions=1 {
  \newcommand\tab[1][1cm]{\hspace*{#1}}
\begin{answer}
\\ \\
$D_{KL}(P(X,Y)||Q(X, Y)) \ = \ \sum_{x, y} P(x, y)log\frac{P(x,y)}{Q(x,y)}$ \\ \\
\tab[3cm] $ = \ \sum_{x, y} P(x, y)log\frac{P(x)P(y|x)}{Q(x)Q(y|x)}$ \\ \\
\tab[3cm] $ = \ \sum_{x, y} P(x, y)log\frac{P(x)}{Q(x)} +P(x, y)log\frac{P(y|x)}{Q(y|x)} $ \\ \\
\tab[3cm] $ = \ \sum_{x} P(x, y)log\frac{P(x)}{Q(x)} + \sum_{x}P(x)\sum_{y}P(y|x)log\frac{P(y|x)}{Q(y|x)} $ \\ \\
\tab[3cm] $ = D_{KL}(P(X)||Q(X)) + D_{KL}(P(Y|X)||Q(Y|X)) $ \\ \\
\end{answer}

} \fi

  \item\subquestionpoints{5} \textbf{KL and maximum likelihood.}

Consider a density estimation problem, and suppose we are given a training set $\{\xsi; i=1,\ldots, \nexp\}$.  Let the empirical distribution be $\hat{P}(x) = \frac{1}{\nexp}\sum_{i=1}^{\nexp} 1\{\xsi=x\}$. ($\hat{P}$ is just the uniform distribution over the training set; i.e., sampling from the empirical distribution is the same as picking a random example from the training set.)

Suppose we have some family of distributions $P_\theta$ parameterized by $\theta$.
(If you like, think of $P_\theta(x)$ as an alternative notation for $P(x;\theta)$.)
Prove that finding the maximum likelihood estimate for the parameter $\theta$ is equivalent to finding $P_{\theta}$ with minimal KL divergence from $\hat{P}$. I.e. prove:
\[
\arg\min_\theta \KL(\hat{P}\|P_\theta)
= \arg\max_\theta \sum_{i=1}^\nexp \log P_\theta(\xsi)
\]

{\bf Remark.} Consider the relationship between parts (b-c) and multi-variate Bernoulli Naive Bayes parameter estimation. In the Naive Bayes model we assumed $P_\theta$ is of the following form: $P_{\theta}(x,y) = p(y)\prod_{i=1}^{\di} p(x_i|y)$.  By the chain rule for KL divergence, we therefore have:
$$
\KL(\hat{P}\|P_\theta)
= \KL(\hat{P}(y)\|p(y)) + \sum_{i=1}^{\di} \KL(\hat{P}(x_i|y)\|p(x_i|y)).
$$
This shows that finding the maximum likelihood/minimum KL-divergence estimate of the parameters decomposes into $2n+1$ independent optimization problems: One for the class priors $p(y)$, and one for each of the conditional distributions $p(x_i|y)$ for each feature $x_i$ given each of the two possible labels for $y$.  Specifically, finding the maximum likelihood estimates for each of these problems individually results in also maximizing the likelihood of the joint distribution.  (If you know what Bayesian networks are, a similar remark applies to parameter estimation for them.)



\ifnum\solutions=1 {
  \newcommand\tab[1][1cm]{\hspace*{#1}}
\begin{answer}
\\ \\
$argmin_\theta D_{KL}(\hat{P}||P_\theta) = argmin_\theta \sum_x \hat{P}(x)log\hat{P}(x) - \hat{P}(x)logP_\theta(x)$ \\ \\
\tab[3cm]$ = argmin_\theta \sum_x - \hat{P}(x)logP_\theta(x)$ \\ \\
\tab[3cm]$ = argmax_\theta \sum_x \hat{P}(x)logP_\theta(x)$ \\ \\
\tab[3cm]$ = argmax_\theta \frac{1}{n} \sum_x  \sum_{i=1}^n  \{x^i = x\} logP_\theta(x)$ \\ \\
\tab[3cm]$ = argmax_\theta  \sum_{i=1}^n  logP_\theta(x^i)$ \\ \\
\end{answer}

} \fi

\end{enumerate}
